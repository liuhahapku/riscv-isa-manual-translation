\chapter{Control and Status Registers (CSRs) \\ 
\hl{控制和状态寄存器}}
\label{chap:priv-csrs}

The SYSTEM major opcode is used to encode all privileged instructions
in the RISC-V ISA.
These can be divided into two main classes: those that atomically
read-modify-write control and status registers (CSRs), which are defined in
the Zicsr extension, and all other privileged instructions.
The privileged architecture requires the Zicsr extension; which other
privileged instructions are required depends on the privileged-architecture
feature set.

\hl{
    在RISC-V指令集架构中,\bf{系}\bf{统}的操作码主要用来编码特权架构的指令。
    这些指令主要可以分为两类:一类是原子性地读-操作-写控制状态寄存器(CSRs)
    的指令,这部分指令定义在Zicsr扩展中,另一类是其他的特权指令。
    特权架构需要Zicsr扩展,而还需要哪些其他特权指令取决于特权体系架构的特性集。
}

In addition to the user-level
state described in Volume I of this manual, an implementation may
contain additional CSRs, accessible by some subset of the privilege
levels using the CSR instructions described in the user-level manual.
In this chapter, we map out the CSR address space.  The following
chapters describe the function of each of the CSRs according to
privilege level, as well as the other privileged instructions which
are generally closely associated with a particular privilege level.
Note that although CSRs and instructions are associated with one
privilege level, they are also accessible at all higher privilege
levels.

\hl{
  除了本手册第一卷定义的用户状态控制和状态寄存器外,硬件实现可能还包括额外的
  控制和状态寄存器,通过在用户级手册中描述的CSR指令,这些寄存器可以被特权等级
  的某些子集访问。在这章里,我们对控制和状态寄存器的地址空间进行映射和规划。
  接下来的章节我们会按照特权等级的顺序依次描述每一个控制和状态寄存器的作用,
  以及其他通常与特定特权等级密切相关的特权指令。值得注意的是,虽然控制和状态
  寄存器以及相关指令是对应于某个特权等级的,但它们同时可以被所有更高的特权等级访问。
}

\section{CSR Address Mapping Conventions \\ 
  \hl{
    控制和状态寄存器地址空间映射约定
  }
}

The standard RISC-V ISA sets aside a 12-bit encoding space (csr[11:0])
for up to 4,096 CSRs.  By convention, the upper 4 bits of the CSR
address (csr[11:8]) are used to encode the read and write
accessibility of the CSRs according to privilege level as shown in
Table~\ref{csrrwpriv}.  The top two bits (csr[11:10]) indicate whether
the register is read/write ({\tt 00}, {\tt 01}, or {\tt 10}) or
read-only ({\tt 11}).  The next two bits (csr[9:8]) encode the lowest
privilege level that can access the CSR.

\hl{
  标准RSIC-V指令集架构设置了12位的控制和状态寄存器编码空间(csr[11:0]),
  最多可以编码4,096个控制和状态寄存器。依照约定,CSR地址空间的高4位(csr[11:8])
  被用于编码读写权限以及各个特权等级对CSR的访问权限,如表~\ref{csrrwpriv}所示。
  高2位(csr[11:10])表示该寄存器是可读可写的({\tt 00}, {\tt 01}, or {\tt 10})
  还是只读的({\tt 11})。接下来的2位(csr[9:8])编码了可以访问该CSR的最低特权等级。
}

\begin{commentary}
The CSR address convention uses the upper bits of the CSR address to
encode default access privileges.  This simplifies error checking in
the hardware and provides a larger CSR space, but does constrain the
mapping of CSRs into the address space.

Implementations might allow a more-privileged level to trap otherwise
permitted CSR accesses by a less-privileged level to allow these
accesses to be intercepted.  This change should be transparent to the
less-privileged software.
\end{commentary}

\begin{commentary} 
  \hl{
    CSR的地址约定使用CSR地址的高几位来编码默认的可访问等级,这简化了硬件的错误检查
    而且提供了一个更大的地址空间,但确实限制了控制和状态寄存器到地址空间的映射。\\
    \indent 当低特权等级试图访问控制和状态寄存器时,硬件实现也许会让其陷入更高特权等级,以此
    来拦截这些访问。这个变化过程对低特权等级的软件来说应该是透明的。 
  }
\end{commentary}

{
\begin{table*}[h!]
\small
\begin{center}
\begin{tabular}{|c|c|c|c|l|}
\hline
\multicolumn{3}{|c|}{CSR Address} & Hex \hl{十六进制}& \multicolumn{1}{c|}{Use and Accessibility \hl{用途及读写权限}}\\ \cline{1-3}
[11:10] & [9:8] & [7:4]                  &  & \\
\hline
\multicolumn{5}{|c|}{User CSRs \hl{用户级 CSRs}}  \\
\hline
\tt   00   &\tt   00  &\tt   XXXX   & \tt 0x000-0x0FF & Standard read/write \hl{标准可读可写}\\
\tt   01   &\tt   00  &\tt   XXXX   & \tt 0x400-0x4FF & Standard read/write \\
\tt   10   &\tt   00  &\tt   XXXX   & \tt 0x800-0x8FF & Custom read/write \hl{自定义可读可写}\\
\tt   11   &\tt   00  &\tt   0XXX   & \tt 0xC00-0xC7F & Standard read-only \\
\tt   11   &\tt   00  &\tt   10XX   & \tt 0xC80-0xCBF & Standard read-only \\
\tt   11   &\tt   00  &\tt   11XX   & \tt 0xCC0-0xCFF & Custom read-only \\
\hline
\multicolumn{5}{|c|}{Supervisor CSRs \hl{监管者级 CSRs}}  \\
\hline
\tt   00   &\tt   01  &\tt   XXXX   & \tt 0x100-0x1FF & Standard read/write \\
\tt   01   &\tt   01  &\tt   0XXX   & \tt 0x500-0x57F & Standard read/write \\
\tt   01   &\tt   01  &\tt   10XX   & \tt 0x580-0x5BF & Standard read/write \\
\tt   01   &\tt   01  &\tt   11XX   & \tt 0x5C0-0x5FF & Custom read/write \\
\tt   10   &\tt   01  &\tt   0XXX   & \tt 0x900-0x97F & Standard read/write \\
\tt   10   &\tt   01  &\tt   10XX   & \tt 0x980-0x9BF & Standard read/write \\
\tt   10   &\tt   01  &\tt   11XX   & \tt 0x9C0-0x9FF & Custom read/write \\
\tt   11   &\tt   01  &\tt   0XXX   & \tt 0xD00-0xD7F & Standard read-only \\
\tt   11   &\tt   01  &\tt   10XX   & \tt 0xD80-0xDBF & Standard read-only \\
\tt   11   &\tt   01  &\tt   11XX   & \tt 0xDC0-0xDFF & Custom read-only \\
\hline
\multicolumn{5}{|c|}{Hypervisor CSRs \hl{超级监管者级 CSRs}} \\
\hline
\tt   00   &\tt   10  &\tt   XXXX   & \tt 0x200-0x2FF & Standard read/write \\
\tt   01   &\tt   10  &\tt   0XXX   & \tt 0x600-0x67F & Standard read/write \\
\tt   01   &\tt   10  &\tt   10XX   & \tt 0x680-0x6BF & Standard read/write \\
\tt   01   &\tt   10  &\tt   11XX   & \tt 0x6C0-0x6FF & Custom read/write \\
\tt   10   &\tt   10  &\tt   0XXX   & \tt 0xA00-0xA7F & Standard read/write \\
\tt   10   &\tt   10  &\tt   10XX   & \tt 0xA80-0xABF & Standard read/write \\
\tt   10   &\tt   10  &\tt   11XX   & \tt 0xAC0-0xAFF & Custom read/write \\
\tt   11   &\tt   10  &\tt   0XXX   & \tt 0xE00-0xE7F & Standard read-only \\
\tt   11   &\tt   10  &\tt   10XX   & \tt 0xE80-0xEBF & Standard read-only \\
\tt   11   &\tt   10  &\tt   11XX   & \tt 0xEC0-0xEFF & Custom read-only \\
\hline
\multicolumn{5}{|c|}{Machine CSRs \hl{机器级 CSRs}}  \\
\hline
\tt   00   &\tt   11  &\tt   XXXX   & \tt 0x300-0x3FF & Standard read/write \\
\tt   01   &\tt   11  &\tt   0XXX   & \tt 0x700-0x77F & Standard read/write \\
\tt   01   &\tt   11  &\tt   100X   & \tt 0x780-0x79F & Standard read/write \\
\tt   01   &\tt   11  &\tt   1010   & \tt 0x7A0-0x7AF & Standard read/write debug CSRs  \\
\tt   01   &\tt   11  &\tt   1011   & \tt 0x7B0-0x7BF & Debug-mode-only CSRs \hl{仅Debug模式可访问}\\
\tt   01   &\tt   11  &\tt   11XX   & \tt 0x7C0-0x7FF & Custom read/write \\
\tt   10   &\tt   11  &\tt   0XXX   & \tt 0xB00-0xB7F & Standard read/write \\
\tt   10   &\tt   11  &\tt   10XX   & \tt 0xB80-0xBBF & Standard read/write \\
\tt   10   &\tt   11  &\tt   11XX   & \tt 0xBC0-0xBFF & Custom read/write \\
\tt   11   &\tt   11  &\tt   0XXX   & \tt 0xF00-0xF7F & Standard read-only \\
\tt   11   &\tt   11  &\tt   10XX   & \tt 0xF80-0xFBF & Standard read-only \\
\tt   11   &\tt   11  &\tt   11XX   & \tt 0xFC0-0xFFF & Custom read-only \\
\hline
\end{tabular}
\end{center}
\caption{Allocation of RISC-V CSR address ranges.}
\label{csrrwpriv}
\end{table*}
}

Attempts to access a non-existent CSR raise an illegal instruction
exception.  Attempts to access a CSR without appropriate privilege
level or to write a read-only register also raise illegal instruction
exceptions.  A read/write register might also contain some bits that
are read-only, in which case writes to the read-only bits are ignored.

\hl{
  试图访问不存在的CSR会引起一个非法指令异常。
  试图在不适当的特权等级下访问CSR或者写一个只读的CSR同样会引起非法指令异常。
  一个可读可写的寄存器可能同时含有一些只读的位,这时写这些只读位的操作将被忽略。
}

Table~\ref{csrrwpriv} also indicates the convention to allocate CSR
addresses between standard and custom uses.  The CSR addresses
designated for custom uses will not be redefined by future
standard extensions.

\hl{
  表~\ref{csrrwpriv}描述了标准的或者自定义的CSR地址空间分配的约定。
  被指定为自定义使用的地址空间将来不会被重新用于定义标准扩展。
}

Machine-mode standard read-write CSRs {\tt 0x7A0}--{\tt 0x7BF} are reserved
for use by the debug system.  Of these CSRs, {\tt 0x7A0}--{\tt 0x7AF} are
accessible to machine mode, whereas {\tt 0x7B0}--{\tt 0x7BF} are only visible
to debug mode.  Implementations should raise illegal instruction exceptions on
machine-mode access to the latter set of registers.

\hl{
  机器级标准读写CSRs{\tt 0x7A0}--{\tt 0x7BF}是为调试系统保留的。其中,{\tt 0x7A0}--{\tt 0x7AF}
  可以再机器模式下访问,然而{\tt 0x7B0}--{\tt 0x7BF}只在调试模式下可见。当试图在机器模式下访问
  后面这部分寄存器时,硬件实现应该引起非法指令异常。
}

\begin{commentary}
Effective virtualization requires that as many instructions run natively as
possible inside a virtualized environment, while any privileged accesses trap
to the virtual machine monitor~\cite{goldbergvm}.  CSRs that are read-only at
some lower privilege level are shadowed into separate CSR addresses if they
are made read-write at a higher privilege level.  This avoids trapping
permitted lower-privilege accesses while still causing traps on illegal
accesses.  Currently, the counters are the only shadowed CSRs.
\end{commentary}

\begin{commentary}
  \hl{
    高效的虚拟化要求尽可能多的指令在虚拟环境内部本地运行,然而任何特权访问都会陷入到虚拟机
    监视器中~\cite{goldbergvm}。在低特权等级只读的CSRs如果在高特权等级下是可读可写的,
    那么它们将被投影到另外的CSR地址。这样可以允许的低特权等级的合法访问不产生陷入,同时
    仍然对非法访问产生陷入。当前,只有计数器寄存器是投影CSR。
  }
\end{commentary}


\section{CSR Listing \\ 
\hl{
  CSR 列表
}}

Tables~\ref{ucsrnames}--\ref{mcsrnames1} list the CSRs that have
currently been allocated CSR addresses.  The timers, counters, and
floating-point CSRs are standard user-level CSRs, as well as the
additional user trap registers added by the N extension.  The other
registers are used by privileged code, as described in the following
chapters.  Note that not all registers are required on all
implementations.

\hl{
  表~\ref{ucsrnames}--\ref{mcsrnames1}列出了现在已经被分配了地址的CSRs。
定时器、计数器和浮点CSRs以及由N扩展引入的用户陷入寄存器是标准的用户级CSRs。
其他的寄存器被高特权等级的代码使用,正如后续章节的描述。需要注意,不是所有的
硬件实现都需要所有这些寄存器。
}

\begin{table}[htb!]
\begin{center}
\begin{tabular}{|l|l|l|l|}
\hline
Number    & Privilege & Name & Description \\
          & \hl{访问等级与} &  &  \\
          & \hl{读写权限} &  &  \\
\hline
\multicolumn{4}{|c|}{User Trap Setup \hl{用户陷阱设置}} \\
\hline
\tt 0x000 & URW  &\tt ustatus    & User status register. \hl{用户状态寄存器}\\
\tt 0x004 & URW  &\tt uie        & User interrupt-enable register. \hl{用户中断使能寄存器}\\
\tt 0x005 & URW  &\tt utvec      & User trap handler base address. \hl{用户中断处理程序基地址}\\
\hline
\multicolumn{4}{|c|}{User Trap Handling \hl{用户陷阱处理}} \\
\hline
\tt 0x040 & URW  &\tt uscratch   & Scratch register for user trap handlers. \hl{暂存寄存器}\\
\tt 0x041 & URW  &\tt uepc       & User exception program counter. \hl{用户异常pc}\\
\tt 0x042 & URW  &\tt ucause     & User trap cause. \hl{用户陷阱原因}\\
\tt 0x043 & URW  &\tt utval      & User bad address or instruction. \hl{用户错误地址}\\
\tt 0x044 & URW  &\tt uip        & User interrupt pending. \hl{用户中断未决}\\
\hline
\multicolumn{4}{|c|}{User Floating-Point CSRs} \\
\hline
\tt 0x001 & URW  &\tt fflags     & Floating-Point Accrued Exceptions. \hl{浮点累计异常}\\
\tt 0x002 & URW  &\tt frm        & Floating-Point Dynamic Rounding Mode. \hl{浮点动态舍入模式}\\
\tt 0x003 & URW  &\tt fcsr       & Floating-Point Control and Status
Register ({\tt frm} + {\tt fflags}). \\
&  &  &  \hl{浮点控制和状态寄存器({\tt frm} + {\tt fflags})}\\
\hline
\multicolumn{4}{|c|}{User Counter/Timers \hl{用户计数器/定时器}} \\
\hline
\tt 0xC00 & URO  &\tt cycle         & Cycle counter for RDCYCLE instruction. \\
&  &  &  \hl{周期计数器,记录硬件线程在处理器核心}\\
&  &  &  \hl{上执行的时钟周期数,可被RDCYCLE指令读取}\\
\tt 0xC01 & URO  &\tt time          & Timer for RDTIME instruction. \hl{实时时钟可被RDTIME指令读取}\\
\tt 0xC02 & URO  &\tt instret       & Instructions-retired counter for RDINSTRET instruction. \\
&  &  &  \hl{退休指令计数器,可被RDINSTRET指令读取}\\
\tt 0xC03 & URO  &\tt hpmcounter3   & Performance-monitoring counter. \\
\tt 0xC04 & URO  &\tt hpmcounter4   & Performance-monitoring counter. \\
& & \multicolumn{1}{c|}{\vdots} & \ \\
\tt 0xC1F & URO  &\tt hpmcounter31  & Performance-monitoring counter. \hl{提供可编程事件计数}\\
\tt 0xC80 & URO  &\tt cycleh        & Upper 32 bits of {\tt cycle}, RV32 only. \hl{RV32I中使用,只读上半部}\\
\tt 0xC81 & URO  &\tt timeh         & Upper 32 bits of {\tt time}, RV32 only. \\
\tt 0xC82 & URO  &\tt instreth      & Upper 32 bits of {\tt instret}, RV32 only. \\
\tt 0xC83 & URO  &\tt hpmcounter3h  & Upper 32 bits of {\tt hpmcounter3}, RV32 only. \\
\tt 0xC84 & URO  &\tt hpmcounter4h  & Upper 32 bits of {\tt hpmcounter4}, RV32 only. \\
& & \multicolumn{1}{c|}{\vdots} & \ \\
\tt 0xC9F & URO  &\tt hpmcounter31h & Upper 32 bits of {\tt hpmcounter31}, RV32 only. \\
\hline
\end{tabular}
\end{center}
\caption{Currently allocated RISC-V user-level CSR addresses. \\
 \hl{现已分配空间的RISC-V用户级CSR地址}}
\label{ucsrnames}
\end{table}

\begin{table}[htb!]
\begin{center}
\begin{tabular}{|l|l|l|l|}
\hline
Number    & Privilege & Name & Description \\
\hline
\multicolumn{4}{|c|}{Supervisor Trap Setup\hl{监管者陷阱设置}} \\
\hline
\tt 0x100 & SRW  &\tt sstatus    & Supervisor status register. \\
&  &  &  \hl{监管者状态寄存器}\\
\tt 0x102 & SRW  &\tt sedeleg    & Supervisor exception delegation register. \\
&  &  &  \hl{监管者异常代理寄存器}\\
\tt 0x103 & SRW  &\tt sideleg    & Supervisor interrupt delegation register. \\
&  &  &  \hl{监管者中断代理寄存器}\\
\tt 0x104 & SRW  &\tt sie        & Supervisor interrupt-enable register. \\
&  &  &  \hl{监管者中断使能寄存器}\\
\tt 0x105 & SRW  &\tt stvec      & Supervisor trap handler base address. \\
&  &  &  \hl{监管者陷阱处理程序基地址寄存器}\\
\tt 0x106 & SRW  &\tt scounteren & Supervisor counter enable. \\
&  &  &  \hl{监管者计数器使能寄存器}\\
\hline
\multicolumn{4}{|c|}{Supervisor Trap Handling \hl{监管者陷阱处理}} \\
\hline
\tt 0x140 & SRW  &\tt sscratch   & Scratch register for supervisor trap handlers. \\
&  &  &  \hl{监管者陷阱处理程序地址暂存寄存器}\\
\tt 0x141 & SRW  &\tt sepc       & Supervisor exception program counter. \\
&  &  &  \hl{监管者异常pc}\\
\tt 0x142 & SRW  &\tt scause     & Supervisor trap cause. \\
&  &  &  \hl{监管者陷阱原因寄存器}\\
\tt 0x143 & SRW  &\tt stval      & Supervisor bad address or instruction. \\
&  &  &  \hl{监管者非法地址或指令寄存器}\\
\tt 0x144 & SRW  &\tt sip        & Supervisor interrupt pending. \\
&  &  &  \hl{监管者中断未决}\\
\hline
\multicolumn{4}{|c|}{Supervisor Protection and Translation \hl{监管者保护和地址转换}} \\
\hline
\tt 0x180 & SRW  &\tt satp       & Supervisor address translation and protection. \\
&  &  &  \hl{监管者地址转换和保护寄存器}\\
\hline
\multicolumn{4}{|c|}{Debug/Trace Registers \hl{调试和追踪寄存器}} \\
\hline
\tt 0x5A8 & SRW &\tt scontext & Supervisor-mode context register. \\
&  &  &  \hl{监管者模式上下文寄存器}\\
\hline
\end{tabular}
\end{center}
\caption{Currently allocated RISC-V supervisor-level CSR addresses. \\
\hl{现已分配空间的RISC-V监管者级CSR地址
}}
\label{scsrnames}
\end{table}

\begin{table}[htb!]
\begin{center}
\begin{tabular}{|l|l|l|l|}
\hline
Number    & Privilege & Name & Description \\
\hline
\multicolumn{4}{|c|}{Hypervisor Trap Setup\hl{超级监管者陷阱设置}} \\
\hline
\hline
\tt 0x600 & HRW  &\tt hstatus    & Hypervisor status register. \\
\tt 0x602 & HRW  &\tt hedeleg    & Hypervisor exception delegation register. \\
\tt 0x603 & HRW  &\tt hideleg    & Hypervisor interrupt delegation register. \\
\tt 0x604 & HRW  &\tt hie        & Hypervisor interrupt-enable register. \\
\tt 0x606 & HRW  &\tt hcounteren & Hypervisor counter enable. \\
\tt 0x607 & HRW  &\tt hgeie      & Hypervisor guest external interrupt-enable register. \\
 & & & \hl{超级监管者客户外部中断使能寄存器} \\
\hline
\multicolumn{4}{|c|}{Hypervisor Trap Handling  \hl{超级监管者中断处理} } \\
\hline
\tt 0x643 & HRW  &\tt htval      & Hypervisor bad guest physical address. \\
& & & \hl{超级监管者错误用户物理地址,即陷阱地址} \\
\tt 0x644 & HRW  &\tt hip        & Hypervisor interrupt pending. \\
\tt 0x645 & HRW  &\tt hvip       & Hypervisor virtual interrupt pending. \\
\tt 0x64A & HRW  &\tt htinst     & Hypervisor trap instruction (transformed). \\
\tt 0xE12 & HRO  &\tt hgeip      & Hypervisor guest external interrupt pending. \\
& & & \hl{超级监管者客户外部中断未决} \\
\hline
\multicolumn{4}{|c|}{Hypervisor Protection and Translation} \\
\hline
\tt 0x680 & HRW  &\tt hgatp      & Hypervisor guest address translation and protection. \\
& & & \hl{超级监管者客户地址转换和保护} \\
\hline
\multicolumn{4}{|c|}{Debug/Trace Registers} \\
\hline
\tt 0x6A8 & HRW &\tt hcontext & Hypervisor-mode context register. \\
& & & \hl{超级监管者模式上下文寄存器} \\
\hline
\multicolumn{4}{|c|}{Hypervisor Counter/Timer Virtualization Registers} \\
\hline
\tt 0x605 & HRW  &\tt htimedelta   & Delta for VS/VU-mode timer. \hl{时间差寄存器} \\
\tt 0x615 & HRW  &\tt htimedeltah  & Upper 32 bits of {\tt htimedelta}, RV32 only. \\
\hline
\multicolumn{4}{|c|}{Virtual Supervisor Registers} \\
\hline
\tt 0x200 & HRW  &\tt vsstatus   & Virtual supervisor status register. \\
\tt 0x204 & HRW  &\tt vsie       & Virtual supervisor interrupt-enable register. \\
\tt 0x205 & HRW  &\tt vstvec     & Virtual supervisor trap handler base address. \\
\tt 0x240 & HRW  &\tt vsscratch  & Virtual supervisor scratch register. \\
\tt 0x241 & HRW  &\tt vsepc      & Virtual supervisor exception program counter. \\
\tt 0x242 & HRW  &\tt vscause    & Virtual supervisor trap cause. \\
\tt 0x243 & HRW  &\tt vstval     & Virtual supervisor bad address or instruction. \\
\tt 0x244 & HRW  &\tt vsip       & Virtual supervisor interrupt pending. \\
\tt 0x280 & HRW  &\tt vsatp      & Virtual supervisor address translation and protection. \\
\hline
\end{tabular}
\end{center}
\caption{Currently allocated RISC-V hypervisor-level CSR addresses. \\
\hl{现已分配空间的RISC-V超级监管者级CSR地址
}}
\label{hcsrnames}
\end{table}


\begin{table}[htb!]
\begin{center}
\begin{tabular}{|l|l|l|l|}
\hline
Number    & Privilege & Name & Description \\
\hline
\multicolumn{4}{|c|}{Machine Information Registers \hl{机器信息注册}} \\
\hline
\tt 0xF11 & MRO &\tt mvendorid   & Vendor ID. \hl{制造商ID}\\
\tt 0xF12 & MRO &\tt marchid     & Architecture ID. \hl{指令集架构ID}\\
\tt 0xF13 & MRO &\tt mimpid      & Implementation ID. \hl{硬件实现ID}\\
\tt 0xF14 & MRO &\tt mhartid     & Hardware thread ID. \hl{硬件线程ID}\\
\hline
\multicolumn{4}{|c|}{Machine Trap Setup \hl{机器级陷阱设置}} \\
\hline
\tt 0x300 & MRW  &\tt mstatus    & Machine status register. \hl{机器级状态寄存器}\\
\tt 0x301 & MRW  &\tt misa       & ISA and extensions \hl{ISA及其扩展}\\
\tt 0x302 & MRW  &\tt medeleg    & Machine exception delegation register. \\
\tt 0x303 & MRW  &\tt mideleg    & Machine interrupt delegation register. \\
\tt 0x304 & MRW  &\tt mie        & Machine interrupt-enable register. \\
\tt 0x305 & MRW  &\tt mtvec      & Machine trap-handler base address. \\
\tt 0x306 & MRW  &\tt mcounteren & Machine counter enable. \\
\tt 0x310 & MRW  &\tt mstatush   & Additional machine status register, RV32 only. \\
\hline
\multicolumn{4}{|c|}{Machine Trap Handling \hl{机器级陷阱处理}} \\
\hline
\tt 0x340 & MRW  &\tt mscratch   & Scratch register for machine trap handlers. \\
\tt 0x341 & MRW  &\tt mepc       & Machine exception program counter. \\
\tt 0x342 & MRW  &\tt mcause     & Machine trap cause. \\
\tt 0x343 & MRW  &\tt mtval      & Machine bad address or instruction. \\
\tt 0x344 & MRW  &\tt mip        & Machine interrupt pending. \\
\tt 0x34A & MRW  &\tt mtinst     & Machine trap instruction (transformed). \\
\tt 0x34B & MRW  &\tt mtval2     & Machine bad guest physical address. \\
\hline
\multicolumn{4}{|c|}{Machine Memory Protection \hl{物理内存保护}} \\
\hline
%\tt 0x380 & MRW  &\tt mbase      & Base register. \\
%\tt 0x381 & MRW  &\tt mbound     & Bound register. \\
%\tt 0x382 & MRW  &\tt mibase     & Instruction base register. \\
%\tt 0x383 & MRW  &\tt mibound    & Instruction bound register. \\
%\tt 0x384 & MRW  &\tt mdbase     & Data base register. \\
%\tt 0x385 & MRW  &\tt mdbound    & Data bound register. \\
\tt 0x3A0 & MRW  &\tt pmpcfg0    & Physical memory protection configuration. \\
\tt 0x3A1 & MRW  &\tt pmpcfg1    & Physical memory protection configuration, RV32 only. \\
\tt 0x3A2 & MRW  &\tt pmpcfg2    & Physical memory protection configuration. \\
\tt 0x3A3 & MRW  &\tt pmpcfg3    & Physical memory protection configuration, RV32 only. \\
& & \multicolumn{1}{c|}{\vdots} & \ \\
\tt 0x3AE & MRW  &\tt pmpcfg14   & Physical memory protection configuration. \\
\tt 0x3AF & MRW  &\tt pmpcfg15   & Physical memory protection configuration, RV32 only. \\
\tt 0x3B0 & MRW  &\tt pmpaddr0   & Physical memory protection address register. \\
\tt 0x3B1 & MRW  &\tt pmpaddr1   & Physical memory protection address register. \\
& & \multicolumn{1}{c|}{\vdots} & \ \\
\tt 0x3EF & MRW  &\tt pmpaddr63  & Physical memory protection address register. \\
\hline
\end{tabular}
\end{center}
\caption{Currently allocated RISC-V machine-level CSR addresses. \\ 
\hl{现已分配空间的RISC-V机器级CSR地址
}}
\label{mcsrnames0}
\end{table}

\begin{table}[htb!]
\begin{center}
\begin{tabular}{|l|l|l|l|}
\hline
Number    & Privilege & Name & Description \\
\hline
\multicolumn{4}{|c|}{Machine Counter/Timers \hl{机器计数器/计时器}} \\
\hline
\tt 0xB00 & MRW  &\tt mcycle         & Machine cycle counter. \\
\tt 0xB02 & MRW  &\tt minstret       & Machine instructions-retired counter. \\
\tt 0xB03 & MRW  &\tt mhpmcounter3   & Machine performance-monitoring counter. \\
\tt 0xB04 & MRW  &\tt mhpmcounter4   & Machine performance-monitoring counter. \\
& & \multicolumn{1}{c|}{\vdots} & \ \\
\tt 0xB1F & MRW  &\tt mhpmcounter31  & Machine performance-monitoring counter. \\
\tt 0xB80 & MRW  &\tt mcycleh        & Upper 32 bits of {\tt mcycle}, RV32 only. \\
\tt 0xB82 & MRW  &\tt minstreth      & Upper 32 bits of {\tt minstret}, RV32 only. \\
\tt 0xB83 & MRW  &\tt mhpmcounter3h  & Upper 32 bits of {\tt mhpmcounter3}, RV32 only. \\
\tt 0xB84 & MRW  &\tt mhpmcounter4h  & Upper 32 bits of {\tt mhpmcounter4}, RV32 only. \\
& & \multicolumn{1}{c|}{\vdots} & \ \\
\tt 0xB9F & MRW  &\tt mhpmcounter31h & Upper 32 bits of {\tt mhpmcounter31}, RV32 only. \\
\hline
\multicolumn{4}{|c|}{Machine Counter Setup \hl{机器计数器设置}} \\
\hline
\tt 0x320 & MRW  &\tt mcountinhibit  & Machine counter-inhibit register. \hl{机器计数器禁用寄存器}\\
\tt 0x323 & MRW  &\tt mhpmevent3     & Machine performance-monitoring event selector. \\
\tt 0x324 & MRW  &\tt mhpmevent4     & Machine performance-monitoring event selector. \\
& & \multicolumn{1}{c|}{\vdots} & \hl{机器性能监控事件选择器} \\
\tt 0x33F & MRW  &\tt mhpmevent31    & Machine performance-monitoring event selector. \\
\hline
\multicolumn{4}{|c|}{Debug/Trace Registers (shared with Debug Mode)} \\
\hline
\tt 0x7A0 & MRW &\tt tselect & Debug/Trace trigger register select. \\
\tt 0x7A1 & MRW &\tt tdata1 & First Debug/Trace trigger data register. \\
\tt 0x7A2 & MRW &\tt tdata2 & Second Debug/Trace trigger data register. \\
\tt 0x7A3 & MRW &\tt tdata3 & Third Debug/Trace trigger data register. \\
\tt 0x7A8 & MRW &\tt mcontext & Machine-mode context register. \\
\hline
\multicolumn{4}{|c|}{Debug Mode Registers } \\
\hline
\tt 0x7B0 & DRW &\tt dcsr & Debug control and status register. \\
\tt 0x7B1 & DRW &\tt dpc & Debug PC. \\
\tt 0x7B2 & DRW &\tt dscratch0 & Debug scratch register 0. \\
\tt 0x7B3 & DRW &\tt dscratch1 & Debug scratch register 1. \\
\hline
\end{tabular}
\end{center}
\caption{Currently allocated RISC-V machine-level CSR addresses. \\
\hl{现已分配空间的RISC-V机器级CSR地址
}}
\label{mcsrnames1}
\end{table}

\clearpage

\section{CSR Field Specifications \\
\hl{寄存器字段说明}}


The following definitions and abbreviations are used in specifying the
behavior of fields within the CSRs.

\hl{下面的定义和缩写用于规定控制和状态寄存器中各个字段的行为。}

\subsection*{Reserved Writes Preserve Values, Reads Ignore Values (WPRI)\\
\hl{写时保留不改变,读时忽略(WPRI)}}

Some whole read/write fields are reserved for future use.  Software
should ignore the values read from these fields, and should preserve
the values held in these fields when writing values to other fields of
the same register.
For forward compatibility, implementations that do not furnish these fields
must hardwire them to zero.
These fields are labeled \wpri\ in the register descriptions.

\begin{commentary}
To simplify the software model, any backward-compatible future
definition of previously reserved fields within a CSR must cope with
the possibility that a non-atomic read/modify/write sequence is used
to update other fields in the CSR.  Alternatively, the original CSR
definition must specify that subfields can only be updated atomically,
which may require a two-instruction clear bit/set bit sequence in
general that can be problematic if intermediate values are not legal.
\end{commentary}

\begin{commentary}
  \hl{
    为了简化软件模型,CSR中先前保留字段的任何向后兼容的未来的定义都必须应对
    使用非原子性的读/修改/写来更新CSR中其他字段的可能性。或者,原始的CSR定义
    必须指定子字段只能被原子性地更新,这通常可能需要一个双指令的清除位/设置位的
    序列,如果双指令的中间值不合法,则会产生问题。
  }
\end{commentary}

\subsection*{Write/Read Only Legal Values (WLRL) \\
\hl{只能读写合法值(WLRL}}

Some read/write CSR fields specify behavior for only a subset of
possible bit encodings, with other bit encodings reserved.  Software
should not write anything other than legal values to such a field, and
should not assume a read will return a legal value unless the last
write was of a legal value, or the register has not been written since
another operation (e.g., reset) set the register to a legal value.
These fields are labeled \wlrl\ in the register descriptions.

\hl{
  一些读/写CSR字段仅为可能的位编码子集指定行为,其他位编码保留。软件不应该
  向这样一个字段写入任何合法值以外的内容,并且不应该假设读取将返回一个合法值,
  除非最后一次写入是合法值,或者寄存器自从另一个操作(例如,重置)将寄存器
  设置为合法值后没有被写入。这些字段在寄存器描述中被标记为WLRL。
}

\begin{commentary}
Hardware implementations need only implement enough state bits to
differentiate between the supported values, but must always return the
complete specified bit-encoding of any supported value when read.
\end{commentary}

Implementations are permitted but not required to raise an illegal
instruction exception if an instruction attempts to write a
non-supported value to a \wlrl\ field.  Implementations can
return arbitrary bit patterns on the read of a \wlrl\ field when the last
write was of an illegal value, but the value returned should
deterministically depend on the illegal written value and
the value of the field prior to the write.

\subsection*{Write Any Values, Reads Legal Values (WARL) \\
\hl{
  可写任意值,只能读出合法值(WARL)
}}

Some read/write CSR fields are only defined for a subset of bit
encodings, but allow any value to be written while guaranteeing to
return a legal value whenever read.  Assuming that writing the CSR has
no other side effects, the range of supported values can be determined
by attempting to write a desired setting then reading to see if the
value was retained.  These fields are labeled \warl\ in the register
descriptions.

Implementations will not raise an exception on writes of unsupported
values to a \warl\ field.  Implementations can
return any legal value on the read of a \warl\ field when the last
write was of an illegal value, but the legal value returned should
deterministically depend on the illegal written value and
the architectural state of the hart.

\section{CSR Width Modulation \\
\hl{CSR宽度调整策略}}
\label{sec:csrwidthmodulation}

If the width of a CSR is changed (for example, by changing MXLEN or UXLEN, as
described in Section~\ref{xlen-control}), the values of the {\em writable}
fields and bits of the new-width CSR are, unless specified otherwise,
determined from the previous-width CSR as though by this algorithm:

\begin{enumerate}

\item The value of the previous-width CSR is copied to a temporary register of
the same width.

\item For the read-only bits of the previous-width CSR, the bits at the same
positions in the temporary register are set to zeros.

\item The width of the temporary register is changed to the new width. If the
new width $W$ is narrower than the previous width, the least-significant $W$
bits of the temporary register are retained and the more-significant bits are
discarded. If the new width is wider than the previous width, the temporary
register is zero-extended to the wider width.

\item Each writable field of the new-width CSR takes the value of the bits at
the same positions in the temporary register.

\end{enumerate}

Changing the width of a CSR is not a read or write of the CSR and thus
does not trigger any side effects.


