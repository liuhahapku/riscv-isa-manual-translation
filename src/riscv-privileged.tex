%=======================================================================
% riscv-privileged.tex
%-----------------------------------------------------------------------

\documentclass[twoside,11pt]{book}

% Fix copy/pasting of ligatures in Acrobat
%\input{glyphtounicode.tex}
%\pdfgentounicode=1 %

\input{preamble}
% Chinese support
\usepackage{ctex}

% background color support
\newcommand\hl{\bgroup\markoverwith
  {\textcolor{yellow}{\rule[-.5ex]{2pt}{2.5ex}}}\ULon}

\newcommand\cm{\bgroup\markoverwith
  {\textcolor{green}{\rule[-.5ex]{2pt}{2.5ex}}}\ULon}

\newcommand{\privrev}{1.12-draft}
\newcommand{\privmonthyear}{June 2019}

\newcommand{\commentmonthyear}{2021 年 6 月}
\newcommand{\commentor}{刘孝男}

\setcounter{secnumdepth}{3}
\setcounter{tocdepth}{3}

\begin{document}

\title{\vspace{-0.7in}\Large {\bf The RISC-V Instruction Set Manual\\ 
\hl{RISC-V指令集手册} } \\
  \large {\bf Volume II: Privileged Architecture\\
  \hl{卷二:特权架构}} \\
  Document Version \privrev \\
  \hl{版本 \privrev}
  \vspace{-0.1in}}

\author{Editors: Andrew Waterman$^{1}$, Krste Asanovi\'{c}$^{1,2}$, John Hauser \\
  $^{1}$SiFive Inc., \\
  $^{2}$CS Division, EECS Department, University of California, Berkeley \\
  {\tt andrew@sifive.com, krste@berkeley.edu, jh.riscv@jhauser.us} \\
  \privmonthyear \\
  \cm{翻译与评注:\commentor} \\
  \cm{邮箱:liuxiaonan@eswin.com} \\
  \cm{时间: \commentmonthyear}
}

\date{} 
\maketitle

RISC-V 特权架构的实现与操作系统等系统软件密切相关。笔者在互联网上没有看到满意的
RISC-V 特权架构相关中文资料,遂决定开始翻译并批注 RISC-V 特权架构手册,
希望能给团队提供关于 RISC-V 硬件如何对 OS 及系统软件提供支持的清晰认识。
该工作的成果开源后,也可作为 ESWIN 在 RISC-V 业界做出的一小点贡献。

文中的翻译部分会用\hl{亮黄色}标出,评注部分会用\cm{绿色}标出。
为节约时间,笔者只翻译和评注自认为比较重要和难懂的部分,有不足之处,待日后继续完善。\\ 
   
\huge{翻译与评注工作记录} \\
\normalsize
   \\ 
   2021年6月7日\\
  开始翻译与评注,完成了第一章 introdution 的翻译。\\
   2021年6月9日\\
  完成第二章 priv-csrs 的翻译,以及第三章的前面一小部分的翻译 \\

\newpage

Contributors to all versions of the spec in alphabetical order (please contact
editors to suggest corrections): Krste Asanovi\'{c}, Peter Ashenden, Rimas
Avi\v{z}ienis, Jacob Bachmeyer, Allen J. Baum, Jonathan Behrens, Paolo Bonzini, Ruslan Bukin,
Christopher Celio, Chuanhua Chang, David Chisnall, Anthony Coulter, Palmer Dabbelt, Monte
Dalrymple, Dennis Ferguson,  Marc Gauthier, Andy Glew,
Gary Guo, Mike Frysinger, John Hauser, David Horner, Olof
Johansson, David Kruckemyer, Yunsup Lee, Andrew Lutomirski, Prashanth Mundkur,
Jonathan Neusch{\"a}fer, Rishiyur
Nikhil, Stefan O'Rear, Albert Ou, John Ousterhout, David Patterson, Dmitri
Pavlov, Kade Phillips, Josh Scheid, Colin Schmidt, Michael Taylor, Wesley Terpstra, Matt Thomas, Tommy Thorn, Ray
VanDeWalker, Megan Wachs, Steve Wallach, Andrew Waterman, Clifford Wolf,
and Reinoud Zandijk.

This document is released under a Creative Commons Attribution 4.0
International License.

This document is a derivative of the RISC-V
privileged specification version 1.9.1 released under following license:
\copyright \,2010--2017 Andrew Waterman, Yunsup Lee, Rimas
Avi\v{z}ienis, David Patterson, Krste Asanovi\'{c}. 
Creative Commons Attribution 4.0 International License.

Please cite as: ``The RISC-V Instruction Set
Manual, Volume II: Privileged Architecture, Document Version \privrev'', Editors
Andrew Waterman and Krste Asanovi\'{c}, RISC-V Foundation, \privmonthyear.

\markboth{Volume II: RISC-V Privileged Architectures V\privrev}
{Volume II: RISC-V Privileged Architectures V\privrev}
\thispagestyle{empty}

\frontmatter

%\input{priv-preface}

{\hypersetup{linktoc=all,hidelinks}
\tableofcontents
}

\mainmatter

\chapter{Introduction}

This document describes the RISC-V privileged architecture, which
covers all aspects of RISC-V systems beyond the unprivileged ISA,
including privileged instructions as well as additional functionality
required for running operating systems and attaching external devices.

\hl{
  本文档描述了RISC-V特权架构,覆盖了RISC-V体系除非特权指令集架构以外的所有方面,
  包括特权指令集以及运行操作系统和连接外部设备所需要的其他功能。
}

\begin{commentary}
Commentary on our design decisions is formatted as in this paragraph,
and can be skipped if the reader is only interested in the
specification itself.
\end{commentary}

\begin{commentary}
  \hl{
    关于我们设计的评论会穿插在段落中,如果读者只对规范本身感兴趣,可以跳过这些评论。
  }
\end{commentary}

\begin{commentary}
We briefly note that the entire privileged-level design described in
this document could be replaced with an entirely different
privileged-level design without changing the unprivileged ISA, and
possibly without even changing the ABI.  In particular, this
privileged specification was designed to run existing popular
operating systems, and so embodies the conventional level-based
protection model.  Alternate privileged specifications could embody
other more flexible protection-domain models.  For simplicity of
expression, the text is written as if this was the only possible
privileged architecture.
\end{commentary}

\begin{commentary}
  \hl{
    这里我们简要说明一下,本规范所描述的整个特权级架构的设计可以在不修改非特权
    ISA的情况下,被另一套不同的特权级架构设计完全替代,甚至不需要修改ABI 。
    另外,由于该特权规范被设计用于运行现有流行的操作系统,因此体现了传统
    的基于不同层次划分的保护模型。其他的特权规范设计可以体现其他更灵活的保护
    域模型。为了简单起见,本文档的写作风格表现得好像这是唯一可能的一种特权架构。
  }
\end{commentary}

\section{RISC-V Privileged Software Stack Terminology ~\\  
\hl{RiSC-V特权软件栈术语} }

This section describes the terminology we use to describe components
of the wide range of possible privileged software stacks for RISC-V. \\
\hl{本节描述了我们用来描述RISC-V各种可能的特权软件栈组件的术语。}

Figure~\ref{fig:privimps} shows some of the possible software stacks
that can be supported by the RISC-V architecture.  The left-hand side
shows a simple system that supports only a single application running
on an application execution environment (AEE).  The application is
coded to run with a particular application binary interface (ABI).
The ABI includes the supported user-level ISA plus a set of ABI calls to
interact with the AEE.  The ABI hides details of the AEE from the
application to allow greater flexibility in implementing the AEE.  The
same ABI could be implemented natively on multiple different host OSs,
or could be supported by a user-mode emulation environment running on
a machine with a different native ISA.

\hl{
  图~\ref{fig:privimps}展示了一些可RISC-V架构支持的软件栈。最左边的图表示
  一个简单系统,仅支持单个应用运行在一个应用执行环境(AEE)中。这个应用程序使用特定的
  应用程序二进制接口(ABI)来编写和运行。这个ABI包括所支持的用户级ISA外加一组与AEE
  交互的ABI调用。ABI对应用程序隐藏了AEE的细节,以便给AEE的实现留下更大的灵活性。
  相同的ABI可以在多个不同的主机操作系统上本地实现,或者可以被一个用户态模拟环境所支持,
  该环境运行在拥有不同原生ISA的机器上的。
}

\begin{figure}[th]
\centering
\includegraphics[width=\textwidth]{figs/privimps.pdf}
\caption{Different implementation stacks supporting various forms of
  privileged execution.}
\label{fig:privimps}
\end{figure}

\begin{commentary}
Our graphical convention represents abstract interfaces using black
boxes with white text, to separate them from concrete instances of
components implementing the interfaces.
\end{commentary}

\begin{commentary}
  \hl{
    图片中我们约定使用带有白色文本的黑色方框表示抽象接口,以便将它们与实现接口的
    组件的具体实例分开。
  }
\end{commentary}

The middle configuration shows a conventional operating system (OS)
that can support multiprogrammed execution of multiple
applications. Each application communicates over an ABI with the OS,
which provides the AEE.  Just as applications interface with an AEE
via an ABI, RISC-V operating systems interface with a supervisor
execution environment (SEE) via a supervisor binary interface (SBI).
An SBI comprises the user-level and supervisor-level ISA together with
a set of SBI function calls.  Using a single SBI across all SEE
implementations allows a single OS binary image to run on any SEE.
The SEE can be a simple boot loader and BIOS-style IO system in a
low-end hardware platform, or a hypervisor-provided virtual machine in
a high-end server, or a thin translation layer over a host operating
system in an architecture simulation environment.

\hl{
  中间的结构展示了一个传统的支持多应用和多线程的操作系统(OS)。每个应用通过ABI
  与操作系统交互,操作系统提供了应用程序执行环境。正如应用程序通过ABI与AEE交互,
  RISC-V操作系统通过监管者二进制接口(SBI)与监管者执行环境(SEE)进行交互。SBI
  由用户级和监管者级ISA以及一组SBI函数调用组成。在所有的SEE实现中使用同一组SBI
  允许一个OS的二进制映像跑在所有的SEE上。这个SEE可以是低端硬件平台上一个简单的
  boot loader和BIOS风格的IO系统,也可以是高端服务器上超级监管者提供的虚拟机,还可以是
  指令集架构模拟环境中一个主机操作系统提供的薄转换层。
}

\begin{commentary}
Most supervisor-level ISA definitions do not separate the SBI from the
execution environment and/or the hardware platform, complicating
virtualization and bring-up of new hardware platforms.
\end{commentary}

\begin{commentary}
  \hl{
    大多数监管者级ISA规范没有把SBI从执行环境和/或硬件平台分离出来,这使得虚拟化
    以及发展新的硬件平台变得复杂。
  }
\end{commentary}

The rightmost configuration shows a virtual machine monitor
configuration where multiple multiprogrammed OSs are supported by a
single hypervisor.  Each OS communicates via an SBI with the
hypervisor, which provides the SEE.  The hypervisor communicates with
the hypervisor execution environment (HEE) using a hypervisor binary
interface (HBI), to isolate the hypervisor from details of the
hardware platform.

\hl{
  最右边的结构展示了一个虚拟机管理器结构,多个支持多线程的OS由单个超级监管者支持。每个OS通过
  SBI与超级监管者交互,而超级监管者提供了监管者执行环境(SEE)。超级监管者与超级监管者
  执行环境之间通过超级监管者二进制接口(HBI)进行交互,以此来对超级监管者屏蔽硬件
  平台的细节。
}

\begin{commentary}
The ABI, SBI, and HBI are still a work-in-progress, but we are now
prioritizing support for Type-2 hypervisors where the SBI is provided
recursively by an S-mode OS.
\end{commentary}

\begin{commentary}
  \hl{
    ABI,SBI和HBI还在发展过程中,现在我们优先支持了Type-2型超级监管者,
  其中SBI由一个监管者模式的{OS}递归地提供。
  }
\end{commentary}

Hardware implementations of the RISC-V ISA will generally require
additional features beyond the privileged ISA to support the various
execution environments (AEE, SEE, or HEE).

\hl{
  RISC-V ISA的硬件实现通常需要特权ISA之外的其他特性来支持各种执行环境(AEE、SEE或HEE)。
}

\section{Privilege Levels \\ 
\hl{特权等级}}

At any time, a RISC-V hardware thread ({\em hart}) is running at some
privilege level encoded as a mode in one or more CSRs (control and
status registers).  Three RISC-V privilege levels are currently defined
as shown in Table~\ref{privlevels}.

\hl{
  一个RISC-V硬件线程({\em hart})在任意时刻都运行在某个特权等级下,这个等级
  被编码成一个或多个控制和状态寄存器CSR中的一种模式。如表~\ref{privlevels}所示,
  RISC-V现在支持三个特权等级。
}

\begin{table*}[h!]
\begin{center}
\begin{tabular}{|c|c|c|c|}
  \hline
   Level & Encoding & Name      & Abbreviation \\ \hline  
   0     & \tt 00   & User/Application \hl{用户级} & U     \\ 
   1     & \tt 01   & Supervisor \hl{监管者级} & S           \\ 
   2     & \tt 10   & {\em Reserved} \hl{保留}&            \\ 
   3     & \tt 11   & Machine \hl{机器级}   & M           \\ 
  \hline
 \end{tabular}
\end{center}
\caption{RISC-V privilege levels.}
\label{privlevels}
\end{table*}

Privilege levels are used to provide protection between different
components of the software stack, and attempts to perform operations
not permitted by the current privilege mode will cause an exception to
be raised.  These exceptions will normally cause traps into an
underlying execution environment.

\hl{
  特权等级存在的目的是在软件栈的不同成分之间提供保护,执行在当前特权模式下不被允许的
  操作会抛出异常。这些异常通常会造成陷入更底层的执行环境。
}

\begin{commentary}
In the description, we try to separate the privilege level for which
code is written, from the privilege mode in which it runs, although
the two are often tied.  For example, a supervisor-level operating
system can run in supervisor-mode on a system with three privilege
modes, but can also run in user-mode under a classic virtual machine
monitor on systems with two or more privilege modes.  In both cases,
the same supervisor-level operating system binary code can be used,
coded to a supervisor-level SBI and hence expecting to be able to use
supervisor-level privileged instructions and CSRs.  When running a
guest OS in user mode, all supervisor-level actions will be trapped
and emulated by the SEE running in the higher-privilege level.
\end{commentary}

\begin{commentary}
  \hl{
    这里我们尝试将代码本身的特权模式和它运行在哪个特权模式下区分开来,虽然他们俩
    通常是结合在一起的。举个例子,一个监管者级的操作系统能够以监管者模式运行在一个
    拥有三种特权等级的系统上,也能通过一个经典虚拟机运行在用户模式下,而该虚拟机
    运行在拥有两个或以上特权等级的系统上。在上述两种情况下,同一份监管者级操作系统
    的二进制代码能够被编写成监管者级的SBI,因此可以使用监管者特权级别的指令和CSRs。
    当以用户模式运行一个客户操作系统时,所有监管者级别的动作都会陷入并由运行在更高
    特权等级的SEE模拟出来。
  }
\end{commentary}

The machine level has the highest privileges and is the only mandatory
privilege level for a RISC-V hardware platform.  Code run in
machine-mode (M-mode) is usually inherently trusted, as it has
low-level access to the machine implementation.  M-mode can be used to
manage secure execution environments on RISC-V.  User-mode (U-mode)
and supervisor-mode (S-mode) are intended for conventional application
and operating system usage respectively.

\hl{
  机器级是最高的特权等级,也是RISC-V硬件平台唯一必须实现的特权等级。运行在机器模式
  (M-mode)的代码通常是默认合法的,因为它对机器的实现具有底层的访问资格。机器模式
  可用于管理RISC-V上的安全执行环境。用户模式(U-mode)和监管者模式(S-Mode)
  分别用于常规应用程序和操作系统的使用场景。
}

Each privilege level has a core set of privileged ISA extensions with optional
extensions and variants.  For example, machine-mode supports an optional
standard extension for memory protection.  Also, supervisor mode can be
extended to support Type-2 hypervisor execution as described in
Chapter~\ref{hypervisor}.

Implementations might provide anywhere from 1 to 3 privilege modes
trading off reduced isolation for lower implementation cost, as shown
in Table~\ref{privcombs}.

\hl{
  每一种特权级别都有一组核心的特权指令集扩展以及可选的扩展和变体。比如,机器模式支持可选的
  标准内存保护扩展。同样的,监管者模式可以被扩展为支持Type-2型超级监管者的运行,正如第~\ref{hypervisor}
  章所描述的那样。\newline \newline
  通过在低实现成本和低隔离性之间权衡,硬件平台的的具体实现可以提供从1到3级的特权模式,如
  表~\ref{privcombs}所示。
}

\begin{table*}[h!]
\begin{center}
\begin{tabular}{|c|l|l|}
  \hline
   Number of levels &  Supported Modes & Intended Usage \\ \hline  
   1     & M          & Simple embedded systems \\ 
        &           & \hl{简单嵌入式系统} \\ 
   2     & M, U       & Secure embedded systems \\ 
   &           & \hl{安全嵌入式系统} \\ 
   3     & M, S, U    & Systems running Unix-like operating systems\\ 
   &           & \hl{运行类Unix操作系统的系统} \\ 
  \hline
 \end{tabular}
\end{center}
\caption{Supported combinations of privilege modes. \hl{受到支持的特权模式组合}}
\label{privcombs}
\end{table*}

All hardware implementations must provide M-mode, as this is the only
mode that has unfettered access to the whole machine.  The simplest
RISC-V implementations may provide only M-mode, though this will
provide no protection against incorrect or malicious application code.

\hl{
  所有的硬件实现都必须提供机器模式,因为这是唯一一个拥有对整个机器无限制的访问权限
  的模式。最简单的RISC-V实现可能只提供机器模式,虽然这样无法提供针对不正确或恶意
  应用程序代码的保护。
}

\begin{commentary}
  The lock feature of the optional PMP facility can provide some
  limited protection even with only M-mode implemented.
\end{commentary}

\begin{commentary}
  \hl{
    即使只实现了机器模式,可选的物理内存保护(PMP)设施的锁定特性也能提供一些有限的保护。
  }
\end{commentary}

Many RISC-V implementations will also support at least user mode
(U-mode) to protect the rest of the system from application code.
Supervisor mode (S-mode) can be added to provide isolation between a
supervisor-level operating system and the SEE.

\hl{
  许多RISC-V实现至少会提供用户模式(U-mode),来保护系统的其余部分不受应用程序
  代码的影响。还可以添加监管者模式(S-mode),从而在监管者级别的操作系统与操作系统
  执行环境(SEE)之间提供隔离。
}

A hart normally runs application code in U-mode until some trap (e.g.,
a supervisor call or a timer interrupt) forces a switch to a trap
handler, which usually runs in a more privileged mode. The hart will
then execute the trap handler, which will eventually resume execution
at or after the original trapped instruction in U-mode.  Traps that
increase privilege level are termed {\em vertical} traps, while traps
that remain at the same privilege level are termed {\em horizontal}
traps.  The RISC-V privileged architecture provides flexible routing
of traps to different privilege layers.

\hl{
  一个硬件线程(hart)通常在用户模式下运行应用,直到一些陷阱(比如监管者系统调用或者
  时钟中断)强制它跳转到陷阱处理程序,该程序通常运行在更高的特权等级上。接着这个硬件
  线程就会执行这个陷阱处理程序,并最终返回到产生陷阱的用户指令或该指令的下一条指令,
  并恢复用户级的特权等级。能够提升特权等级的陷阱称为垂直陷阱,保持当前特权等级
  的陷阱称为水平陷阱。RISC-V特权架构提供了到不同特权层的灵活的路径。
}

\begin{commentary}
Horizontal traps can be implemented as vertical traps that
return control to a horizontal trap handler in the less-privileged mode.
\end{commentary}

\begin{commentary}
  \hl{
    水平陷阱可以被实现为将控制权交还给更低特权等级的水平陷阱处理程序的垂直陷阱。
  }
\end{commentary}

\section{Debug Mode \\
\hl{调试模式}}

Implementations may also include a debug mode to support off-chip
debugging and/or manufacturing test.  Debug mode (D-mode) can be
considered an additional privilege mode, with even more access than
M-mode. The separate debug specification proposal describes operation
of a RISC-V hart in debug mode.  Debug mode reserves a few CSR
addresses that are only accessible in D-mode, and may also reserve
some portions of the physical address space on a platform.

\hl{
  硬件实现还有可能包含了调试模式,来支持片外调试和流片后的测试。
  调试模式(D-mode)可以认为是一个额外的特权等级,拥有比机器模式更高的访问权限。
  独立的调试规范方案描述了RISC-V核在调试模式下的操作。
  调试模式单独保留了一些控制状态寄存在(CSR)的地址空间,
  这些地址只能在调试模式下访问,
  另外调试模式也可能保留了平台上的一部分物理地址空间。
}

\chapter{Control and Status Registers (CSRs) \\ 
\hl{控制和状态寄存器}}
\label{chap:priv-csrs}

The SYSTEM major opcode is used to encode all privileged instructions
in the RISC-V ISA.
These can be divided into two main classes: those that atomically
read-modify-write control and status registers (CSRs), which are defined in
the Zicsr extension, and all other privileged instructions.
The privileged architecture requires the Zicsr extension; which other
privileged instructions are required depends on the privileged-architecture
feature set.

\hl{
    在RISC-V指令集架构中,\bf{系}\bf{统}的操作码主要用来编码特权架构的指令。
    这些指令主要可以分为两类:一类是原子性地读-操作-写控制状态寄存器(CSRs)
    的指令,这部分指令定义在Zicsr扩展中,另一类是其他的特权指令。
    特权架构需要Zicsr扩展,而还需要哪些其他特权指令取决于特权体系架构的特性集。
}

In addition to the user-level
state described in Volume I of this manual, an implementation may
contain additional CSRs, accessible by some subset of the privilege
levels using the CSR instructions described in the user-level manual.
In this chapter, we map out the CSR address space.  The following
chapters describe the function of each of the CSRs according to
privilege level, as well as the other privileged instructions which
are generally closely associated with a particular privilege level.
Note that although CSRs and instructions are associated with one
privilege level, they are also accessible at all higher privilege
levels.

\hl{
  除了本手册第一卷定义的用户状态控制和状态寄存器外,硬件实现可能还包括额外的
  控制和状态寄存器,通过在用户级手册中描述的CSR指令,这些寄存器可以被特权等级
  的某些子集访问。在这章里,我们对控制和状态寄存器的地址空间进行映射和规划。
  接下来的章节我们会按照特权等级的顺序依次描述每一个控制和状态寄存器的作用,
  以及其他通常与特定特权等级密切相关的特权指令。值得注意的是,虽然控制和状态
  寄存器以及相关指令是对应于某个特权等级的,但它们同时可以被所有更高的特权等级访问。
}

\section{CSR Address Mapping Conventions \\ 
  \hl{
    控制和状态寄存器地址空间映射约定
  }
}

The standard RISC-V ISA sets aside a 12-bit encoding space (csr[11:0])
for up to 4,096 CSRs.  By convention, the upper 4 bits of the CSR
address (csr[11:8]) are used to encode the read and write
accessibility of the CSRs according to privilege level as shown in
Table~\ref{csrrwpriv}.  The top two bits (csr[11:10]) indicate whether
the register is read/write ({\tt 00}, {\tt 01}, or {\tt 10}) or
read-only ({\tt 11}).  The next two bits (csr[9:8]) encode the lowest
privilege level that can access the CSR.

\hl{
  标准RSIC-V指令集架构设置了12位的控制和状态寄存器编码空间(csr[11:0]),
  最多可以编码4,096个控制和状态寄存器。依照约定,CSR地址空间的高4位(csr[11:8])
  被用于编码读写权限以及各个特权等级对CSR的访问权限,如表~\ref{csrrwpriv}所示。
  高2位(csr[11:10])表示该寄存器是可读可写的({\tt 00}, {\tt 01}, or {\tt 10})
  还是只读的({\tt 11})。接下来的2位(csr[9:8])编码了可以访问该CSR的最低特权等级。
}

\begin{commentary}
The CSR address convention uses the upper bits of the CSR address to
encode default access privileges.  This simplifies error checking in
the hardware and provides a larger CSR space, but does constrain the
mapping of CSRs into the address space.

Implementations might allow a more-privileged level to trap otherwise
permitted CSR accesses by a less-privileged level to allow these
accesses to be intercepted.  This change should be transparent to the
less-privileged software.
\end{commentary}

\begin{commentary} 
  \hl{
    CSR的地址约定使用CSR地址的高几位来编码默认的可访问等级,这简化了硬件的错误检查
    而且提供了一个更大的地址空间,但确实限制了控制和状态寄存器到地址空间的映射。\\
    \indent 当低特权等级试图访问控制和状态寄存器时,硬件实现也许会让其陷入更高特权等级,以此
    来拦截这些访问。这个变化过程对低特权等级的软件来说应该是透明的。 
  }
\end{commentary}

{
\begin{table*}[h!]
\small
\begin{center}
\begin{tabular}{|c|c|c|c|l|}
\hline
\multicolumn{3}{|c|}{CSR Address} & Hex \hl{十六进制}& \multicolumn{1}{c|}{Use and Accessibility \hl{用途及读写权限}}\\ \cline{1-3}
[11:10] & [9:8] & [7:4]                  &  & \\
\hline
\multicolumn{5}{|c|}{User CSRs \hl{用户级 CSRs}}  \\
\hline
\tt   00   &\tt   00  &\tt   XXXX   & \tt 0x000-0x0FF & Standard read/write \hl{标准可读可写}\\
\tt   01   &\tt   00  &\tt   XXXX   & \tt 0x400-0x4FF & Standard read/write \\
\tt   10   &\tt   00  &\tt   XXXX   & \tt 0x800-0x8FF & Custom read/write \hl{自定义可读可写}\\
\tt   11   &\tt   00  &\tt   0XXX   & \tt 0xC00-0xC7F & Standard read-only \\
\tt   11   &\tt   00  &\tt   10XX   & \tt 0xC80-0xCBF & Standard read-only \\
\tt   11   &\tt   00  &\tt   11XX   & \tt 0xCC0-0xCFF & Custom read-only \\
\hline
\multicolumn{5}{|c|}{Supervisor CSRs \hl{监管者级 CSRs}}  \\
\hline
\tt   00   &\tt   01  &\tt   XXXX   & \tt 0x100-0x1FF & Standard read/write \\
\tt   01   &\tt   01  &\tt   0XXX   & \tt 0x500-0x57F & Standard read/write \\
\tt   01   &\tt   01  &\tt   10XX   & \tt 0x580-0x5BF & Standard read/write \\
\tt   01   &\tt   01  &\tt   11XX   & \tt 0x5C0-0x5FF & Custom read/write \\
\tt   10   &\tt   01  &\tt   0XXX   & \tt 0x900-0x97F & Standard read/write \\
\tt   10   &\tt   01  &\tt   10XX   & \tt 0x980-0x9BF & Standard read/write \\
\tt   10   &\tt   01  &\tt   11XX   & \tt 0x9C0-0x9FF & Custom read/write \\
\tt   11   &\tt   01  &\tt   0XXX   & \tt 0xD00-0xD7F & Standard read-only \\
\tt   11   &\tt   01  &\tt   10XX   & \tt 0xD80-0xDBF & Standard read-only \\
\tt   11   &\tt   01  &\tt   11XX   & \tt 0xDC0-0xDFF & Custom read-only \\
\hline
\multicolumn{5}{|c|}{Hypervisor CSRs \hl{超级监管者级 CSRs}} \\
\hline
\tt   00   &\tt   10  &\tt   XXXX   & \tt 0x200-0x2FF & Standard read/write \\
\tt   01   &\tt   10  &\tt   0XXX   & \tt 0x600-0x67F & Standard read/write \\
\tt   01   &\tt   10  &\tt   10XX   & \tt 0x680-0x6BF & Standard read/write \\
\tt   01   &\tt   10  &\tt   11XX   & \tt 0x6C0-0x6FF & Custom read/write \\
\tt   10   &\tt   10  &\tt   0XXX   & \tt 0xA00-0xA7F & Standard read/write \\
\tt   10   &\tt   10  &\tt   10XX   & \tt 0xA80-0xABF & Standard read/write \\
\tt   10   &\tt   10  &\tt   11XX   & \tt 0xAC0-0xAFF & Custom read/write \\
\tt   11   &\tt   10  &\tt   0XXX   & \tt 0xE00-0xE7F & Standard read-only \\
\tt   11   &\tt   10  &\tt   10XX   & \tt 0xE80-0xEBF & Standard read-only \\
\tt   11   &\tt   10  &\tt   11XX   & \tt 0xEC0-0xEFF & Custom read-only \\
\hline
\multicolumn{5}{|c|}{Machine CSRs \hl{机器级 CSRs}}  \\
\hline
\tt   00   &\tt   11  &\tt   XXXX   & \tt 0x300-0x3FF & Standard read/write \\
\tt   01   &\tt   11  &\tt   0XXX   & \tt 0x700-0x77F & Standard read/write \\
\tt   01   &\tt   11  &\tt   100X   & \tt 0x780-0x79F & Standard read/write \\
\tt   01   &\tt   11  &\tt   1010   & \tt 0x7A0-0x7AF & Standard read/write debug CSRs  \\
\tt   01   &\tt   11  &\tt   1011   & \tt 0x7B0-0x7BF & Debug-mode-only CSRs \hl{仅Debug模式可访问}\\
\tt   01   &\tt   11  &\tt   11XX   & \tt 0x7C0-0x7FF & Custom read/write \\
\tt   10   &\tt   11  &\tt   0XXX   & \tt 0xB00-0xB7F & Standard read/write \\
\tt   10   &\tt   11  &\tt   10XX   & \tt 0xB80-0xBBF & Standard read/write \\
\tt   10   &\tt   11  &\tt   11XX   & \tt 0xBC0-0xBFF & Custom read/write \\
\tt   11   &\tt   11  &\tt   0XXX   & \tt 0xF00-0xF7F & Standard read-only \\
\tt   11   &\tt   11  &\tt   10XX   & \tt 0xF80-0xFBF & Standard read-only \\
\tt   11   &\tt   11  &\tt   11XX   & \tt 0xFC0-0xFFF & Custom read-only \\
\hline
\end{tabular}
\end{center}
\caption{Allocation of RISC-V CSR address ranges.}
\label{csrrwpriv}
\end{table*}
}

Attempts to access a non-existent CSR raise an illegal instruction
exception.  Attempts to access a CSR without appropriate privilege
level or to write a read-only register also raise illegal instruction
exceptions.  A read/write register might also contain some bits that
are read-only, in which case writes to the read-only bits are ignored.

\hl{
  试图访问不存在的CSR会引起一个非法指令异常。
  试图在不适当的特权等级下访问CSR或者写一个只读的CSR同样会引起非法指令异常。
  一个可读可写的寄存器可能同时含有一些只读的位,这时写这些只读位的操作将被忽略。
}

Table~\ref{csrrwpriv} also indicates the convention to allocate CSR
addresses between standard and custom uses.  The CSR addresses
designated for custom uses will not be redefined by future
standard extensions.

\hl{
  表~\ref{csrrwpriv}描述了标准的或者自定义的CSR地址空间分配的约定。
  被指定为自定义使用的地址空间将来不会被重新用于定义标准扩展。
}

Machine-mode standard read-write CSRs {\tt 0x7A0}--{\tt 0x7BF} are reserved
for use by the debug system.  Of these CSRs, {\tt 0x7A0}--{\tt 0x7AF} are
accessible to machine mode, whereas {\tt 0x7B0}--{\tt 0x7BF} are only visible
to debug mode.  Implementations should raise illegal instruction exceptions on
machine-mode access to the latter set of registers.

\hl{
  机器级标准读写CSRs{\tt 0x7A0}--{\tt 0x7BF}是为调试系统保留的。其中,{\tt 0x7A0}--{\tt 0x7AF}
  可以再机器模式下访问,然而{\tt 0x7B0}--{\tt 0x7BF}只在调试模式下可见。当试图在机器模式下访问
  后面这部分寄存器时,硬件实现应该引起非法指令异常。
}

\begin{commentary}
Effective virtualization requires that as many instructions run natively as
possible inside a virtualized environment, while any privileged accesses trap
to the virtual machine monitor~\cite{goldbergvm}.  CSRs that are read-only at
some lower privilege level are shadowed into separate CSR addresses if they
are made read-write at a higher privilege level.  This avoids trapping
permitted lower-privilege accesses while still causing traps on illegal
accesses.  Currently, the counters are the only shadowed CSRs.
\end{commentary}

\begin{commentary}
  \hl{
    高效的虚拟化要求尽可能多的指令在虚拟环境内部本地运行,然而任何特权访问都会陷入到虚拟机
    监视器中~\cite{goldbergvm}。在低特权等级只读的CSRs如果在高特权等级下是可读可写的,
    那么它们将被投影到另外的CSR地址。这样可以允许的低特权等级的合法访问不产生陷入,同时
    仍然对非法访问产生陷入。当前,只有计数器寄存器是投影CSR。
  }
\end{commentary}


\section{CSR Listing \\ 
\hl{
  CSR 列表
}}

Tables~\ref{ucsrnames}--\ref{mcsrnames1} list the CSRs that have
currently been allocated CSR addresses.  The timers, counters, and
floating-point CSRs are standard user-level CSRs, as well as the
additional user trap registers added by the N extension.  The other
registers are used by privileged code, as described in the following
chapters.  Note that not all registers are required on all
implementations.

\hl{
  表~\ref{ucsrnames}--\ref{mcsrnames1}列出了现在已经被分配了地址的CSRs。
定时器、计数器和浮点CSRs以及由N扩展引入的用户陷入寄存器是标准的用户级CSRs。
其他的寄存器被高特权等级的代码使用,正如后续章节的描述。需要注意,不是所有的
硬件实现都需要所有这些寄存器。
}

\begin{table}[htb!]
\begin{center}
\begin{tabular}{|l|l|l|l|}
\hline
Number    & Privilege & Name & Description \\
          & \hl{访问等级与} &  &  \\
          & \hl{读写权限} &  &  \\
\hline
\multicolumn{4}{|c|}{User Trap Setup \hl{用户陷阱设置}} \\
\hline
\tt 0x000 & URW  &\tt ustatus    & User status register. \hl{用户状态寄存器}\\
\tt 0x004 & URW  &\tt uie        & User interrupt-enable register. \hl{用户中断使能寄存器}\\
\tt 0x005 & URW  &\tt utvec      & User trap handler base address. \hl{用户中断处理程序基地址}\\
\hline
\multicolumn{4}{|c|}{User Trap Handling \hl{用户陷阱处理}} \\
\hline
\tt 0x040 & URW  &\tt uscratch   & Scratch register for user trap handlers. \hl{暂存寄存器}\\
\tt 0x041 & URW  &\tt uepc       & User exception program counter. \hl{用户异常pc}\\
\tt 0x042 & URW  &\tt ucause     & User trap cause. \hl{用户陷阱原因}\\
\tt 0x043 & URW  &\tt utval      & User bad address or instruction. \hl{用户错误地址}\\
\tt 0x044 & URW  &\tt uip        & User interrupt pending. \hl{用户中断未决}\\
\hline
\multicolumn{4}{|c|}{User Floating-Point CSRs} \\
\hline
\tt 0x001 & URW  &\tt fflags     & Floating-Point Accrued Exceptions. \hl{浮点累计异常}\\
\tt 0x002 & URW  &\tt frm        & Floating-Point Dynamic Rounding Mode. \hl{浮点动态舍入模式}\\
\tt 0x003 & URW  &\tt fcsr       & Floating-Point Control and Status
Register ({\tt frm} + {\tt fflags}). \\
&  &  &  \hl{浮点控制和状态寄存器({\tt frm} + {\tt fflags})}\\
\hline
\multicolumn{4}{|c|}{User Counter/Timers \hl{用户计数器/定时器}} \\
\hline
\tt 0xC00 & URO  &\tt cycle         & Cycle counter for RDCYCLE instruction. \\
&  &  &  \hl{周期计数器,记录硬件线程在处理器核心}\\
&  &  &  \hl{上执行的时钟周期数,可被RDCYCLE指令读取}\\
\tt 0xC01 & URO  &\tt time          & Timer for RDTIME instruction. \hl{实时时钟可被RDTIME指令读取}\\
\tt 0xC02 & URO  &\tt instret       & Instructions-retired counter for RDINSTRET instruction. \\
&  &  &  \hl{退休指令计数器,可被RDINSTRET指令读取}\\
\tt 0xC03 & URO  &\tt hpmcounter3   & Performance-monitoring counter. \\
\tt 0xC04 & URO  &\tt hpmcounter4   & Performance-monitoring counter. \\
& & \multicolumn{1}{c|}{\vdots} & \ \\
\tt 0xC1F & URO  &\tt hpmcounter31  & Performance-monitoring counter. \hl{提供可编程事件计数}\\
\tt 0xC80 & URO  &\tt cycleh        & Upper 32 bits of {\tt cycle}, RV32 only. \hl{RV32I中使用,只读上半部}\\
\tt 0xC81 & URO  &\tt timeh         & Upper 32 bits of {\tt time}, RV32 only. \\
\tt 0xC82 & URO  &\tt instreth      & Upper 32 bits of {\tt instret}, RV32 only. \\
\tt 0xC83 & URO  &\tt hpmcounter3h  & Upper 32 bits of {\tt hpmcounter3}, RV32 only. \\
\tt 0xC84 & URO  &\tt hpmcounter4h  & Upper 32 bits of {\tt hpmcounter4}, RV32 only. \\
& & \multicolumn{1}{c|}{\vdots} & \ \\
\tt 0xC9F & URO  &\tt hpmcounter31h & Upper 32 bits of {\tt hpmcounter31}, RV32 only. \\
\hline
\end{tabular}
\end{center}
\caption{Currently allocated RISC-V user-level CSR addresses. \\
 \hl{现已分配空间的RISC-V用户级CSR地址}}
\label{ucsrnames}
\end{table}

\begin{table}[htb!]
\begin{center}
\begin{tabular}{|l|l|l|l|}
\hline
Number    & Privilege & Name & Description \\
\hline
\multicolumn{4}{|c|}{Supervisor Trap Setup\hl{监管者陷阱设置}} \\
\hline
\tt 0x100 & SRW  &\tt sstatus    & Supervisor status register. \\
&  &  &  \hl{监管者状态寄存器}\\
\tt 0x102 & SRW  &\tt sedeleg    & Supervisor exception delegation register. \\
&  &  &  \hl{监管者异常代理寄存器}\\
\tt 0x103 & SRW  &\tt sideleg    & Supervisor interrupt delegation register. \\
&  &  &  \hl{监管者中断代理寄存器}\\
\tt 0x104 & SRW  &\tt sie        & Supervisor interrupt-enable register. \\
&  &  &  \hl{监管者中断使能寄存器}\\
\tt 0x105 & SRW  &\tt stvec      & Supervisor trap handler base address. \\
&  &  &  \hl{监管者陷阱处理程序基地址寄存器}\\
\tt 0x106 & SRW  &\tt scounteren & Supervisor counter enable. \\
&  &  &  \hl{监管者计数器使能寄存器}\\
\hline
\multicolumn{4}{|c|}{Supervisor Trap Handling \hl{监管者陷阱处理}} \\
\hline
\tt 0x140 & SRW  &\tt sscratch   & Scratch register for supervisor trap handlers. \\
&  &  &  \hl{监管者陷阱处理程序地址暂存寄存器}\\
\tt 0x141 & SRW  &\tt sepc       & Supervisor exception program counter. \\
&  &  &  \hl{监管者异常pc}\\
\tt 0x142 & SRW  &\tt scause     & Supervisor trap cause. \\
&  &  &  \hl{监管者陷阱原因寄存器}\\
\tt 0x143 & SRW  &\tt stval      & Supervisor bad address or instruction. \\
&  &  &  \hl{监管者非法地址或指令寄存器}\\
\tt 0x144 & SRW  &\tt sip        & Supervisor interrupt pending. \\
&  &  &  \hl{监管者中断未决}\\
\hline
\multicolumn{4}{|c|}{Supervisor Protection and Translation \hl{监管者保护和地址转换}} \\
\hline
\tt 0x180 & SRW  &\tt satp       & Supervisor address translation and protection. \\
&  &  &  \hl{监管者地址转换和保护寄存器}\\
\hline
\multicolumn{4}{|c|}{Debug/Trace Registers \hl{调试和追踪寄存器}} \\
\hline
\tt 0x5A8 & SRW &\tt scontext & Supervisor-mode context register. \\
&  &  &  \hl{监管者模式上下文寄存器}\\
\hline
\end{tabular}
\end{center}
\caption{Currently allocated RISC-V supervisor-level CSR addresses. \\
\hl{现已分配空间的RISC-V监管者级CSR地址
}}
\label{scsrnames}
\end{table}

\begin{table}[htb!]
\begin{center}
\begin{tabular}{|l|l|l|l|}
\hline
Number    & Privilege & Name & Description \\
\hline
\multicolumn{4}{|c|}{Hypervisor Trap Setup\hl{超级监管者陷阱设置}} \\
\hline
\hline
\tt 0x600 & HRW  &\tt hstatus    & Hypervisor status register. \\
\tt 0x602 & HRW  &\tt hedeleg    & Hypervisor exception delegation register. \\
\tt 0x603 & HRW  &\tt hideleg    & Hypervisor interrupt delegation register. \\
\tt 0x604 & HRW  &\tt hie        & Hypervisor interrupt-enable register. \\
\tt 0x606 & HRW  &\tt hcounteren & Hypervisor counter enable. \\
\tt 0x607 & HRW  &\tt hgeie      & Hypervisor guest external interrupt-enable register. \\
 & & & \hl{超级监管者客户外部中断使能寄存器} \\
\hline
\multicolumn{4}{|c|}{Hypervisor Trap Handling  \hl{超级监管者中断处理} } \\
\hline
\tt 0x643 & HRW  &\tt htval      & Hypervisor bad guest physical address. \\
& & & \hl{超级监管者错误用户物理地址,即陷阱地址} \\
\tt 0x644 & HRW  &\tt hip        & Hypervisor interrupt pending. \\
\tt 0x645 & HRW  &\tt hvip       & Hypervisor virtual interrupt pending. \\
\tt 0x64A & HRW  &\tt htinst     & Hypervisor trap instruction (transformed). \\
\tt 0xE12 & HRO  &\tt hgeip      & Hypervisor guest external interrupt pending. \\
& & & \hl{超级监管者客户外部中断未决} \\
\hline
\multicolumn{4}{|c|}{Hypervisor Protection and Translation} \\
\hline
\tt 0x680 & HRW  &\tt hgatp      & Hypervisor guest address translation and protection. \\
& & & \hl{超级监管者客户地址转换和保护} \\
\hline
\multicolumn{4}{|c|}{Debug/Trace Registers} \\
\hline
\tt 0x6A8 & HRW &\tt hcontext & Hypervisor-mode context register. \\
& & & \hl{超级监管者模式上下文寄存器} \\
\hline
\multicolumn{4}{|c|}{Hypervisor Counter/Timer Virtualization Registers} \\
\hline
\tt 0x605 & HRW  &\tt htimedelta   & Delta for VS/VU-mode timer. \hl{时间差寄存器} \\
\tt 0x615 & HRW  &\tt htimedeltah  & Upper 32 bits of {\tt htimedelta}, RV32 only. \\
\hline
\multicolumn{4}{|c|}{Virtual Supervisor Registers} \\
\hline
\tt 0x200 & HRW  &\tt vsstatus   & Virtual supervisor status register. \\
\tt 0x204 & HRW  &\tt vsie       & Virtual supervisor interrupt-enable register. \\
\tt 0x205 & HRW  &\tt vstvec     & Virtual supervisor trap handler base address. \\
\tt 0x240 & HRW  &\tt vsscratch  & Virtual supervisor scratch register. \\
\tt 0x241 & HRW  &\tt vsepc      & Virtual supervisor exception program counter. \\
\tt 0x242 & HRW  &\tt vscause    & Virtual supervisor trap cause. \\
\tt 0x243 & HRW  &\tt vstval     & Virtual supervisor bad address or instruction. \\
\tt 0x244 & HRW  &\tt vsip       & Virtual supervisor interrupt pending. \\
\tt 0x280 & HRW  &\tt vsatp      & Virtual supervisor address translation and protection. \\
\hline
\end{tabular}
\end{center}
\caption{Currently allocated RISC-V hypervisor-level CSR addresses. \\
\hl{现已分配空间的RISC-V超级监管者级CSR地址
}}
\label{hcsrnames}
\end{table}


\begin{table}[htb!]
\begin{center}
\begin{tabular}{|l|l|l|l|}
\hline
Number    & Privilege & Name & Description \\
\hline
\multicolumn{4}{|c|}{Machine Information Registers \hl{机器信息注册}} \\
\hline
\tt 0xF11 & MRO &\tt mvendorid   & Vendor ID. \hl{制造商ID}\\
\tt 0xF12 & MRO &\tt marchid     & Architecture ID. \hl{指令集架构ID}\\
\tt 0xF13 & MRO &\tt mimpid      & Implementation ID. \hl{硬件实现ID}\\
\tt 0xF14 & MRO &\tt mhartid     & Hardware thread ID. \hl{硬件线程ID}\\
\hline
\multicolumn{4}{|c|}{Machine Trap Setup \hl{机器级陷阱设置}} \\
\hline
\tt 0x300 & MRW  &\tt mstatus    & Machine status register. \hl{机器级状态寄存器}\\
\tt 0x301 & MRW  &\tt misa       & ISA and extensions \hl{ISA及其扩展}\\
\tt 0x302 & MRW  &\tt medeleg    & Machine exception delegation register. \\
\tt 0x303 & MRW  &\tt mideleg    & Machine interrupt delegation register. \\
\tt 0x304 & MRW  &\tt mie        & Machine interrupt-enable register. \\
\tt 0x305 & MRW  &\tt mtvec      & Machine trap-handler base address. \\
\tt 0x306 & MRW  &\tt mcounteren & Machine counter enable. \\
\tt 0x310 & MRW  &\tt mstatush   & Additional machine status register, RV32 only. \\
\hline
\multicolumn{4}{|c|}{Machine Trap Handling \hl{机器级陷阱处理}} \\
\hline
\tt 0x340 & MRW  &\tt mscratch   & Scratch register for machine trap handlers. \\
\tt 0x341 & MRW  &\tt mepc       & Machine exception program counter. \\
\tt 0x342 & MRW  &\tt mcause     & Machine trap cause. \\
\tt 0x343 & MRW  &\tt mtval      & Machine bad address or instruction. \\
\tt 0x344 & MRW  &\tt mip        & Machine interrupt pending. \\
\tt 0x34A & MRW  &\tt mtinst     & Machine trap instruction (transformed). \\
\tt 0x34B & MRW  &\tt mtval2     & Machine bad guest physical address. \\
\hline
\multicolumn{4}{|c|}{Machine Memory Protection \hl{物理内存保护}} \\
\hline
%\tt 0x380 & MRW  &\tt mbase      & Base register. \\
%\tt 0x381 & MRW  &\tt mbound     & Bound register. \\
%\tt 0x382 & MRW  &\tt mibase     & Instruction base register. \\
%\tt 0x383 & MRW  &\tt mibound    & Instruction bound register. \\
%\tt 0x384 & MRW  &\tt mdbase     & Data base register. \\
%\tt 0x385 & MRW  &\tt mdbound    & Data bound register. \\
\tt 0x3A0 & MRW  &\tt pmpcfg0    & Physical memory protection configuration. \\
\tt 0x3A1 & MRW  &\tt pmpcfg1    & Physical memory protection configuration, RV32 only. \\
\tt 0x3A2 & MRW  &\tt pmpcfg2    & Physical memory protection configuration. \\
\tt 0x3A3 & MRW  &\tt pmpcfg3    & Physical memory protection configuration, RV32 only. \\
& & \multicolumn{1}{c|}{\vdots} & \ \\
\tt 0x3AE & MRW  &\tt pmpcfg14   & Physical memory protection configuration. \\
\tt 0x3AF & MRW  &\tt pmpcfg15   & Physical memory protection configuration, RV32 only. \\
\tt 0x3B0 & MRW  &\tt pmpaddr0   & Physical memory protection address register. \\
\tt 0x3B1 & MRW  &\tt pmpaddr1   & Physical memory protection address register. \\
& & \multicolumn{1}{c|}{\vdots} & \ \\
\tt 0x3EF & MRW  &\tt pmpaddr63  & Physical memory protection address register. \\
\hline
\end{tabular}
\end{center}
\caption{Currently allocated RISC-V machine-level CSR addresses. \\ 
\hl{现已分配空间的RISC-V机器级CSR地址
}}
\label{mcsrnames0}
\end{table}

\begin{table}[htb!]
\begin{center}
\begin{tabular}{|l|l|l|l|}
\hline
Number    & Privilege & Name & Description \\
\hline
\multicolumn{4}{|c|}{Machine Counter/Timers \hl{机器计数器/计时器}} \\
\hline
\tt 0xB00 & MRW  &\tt mcycle         & Machine cycle counter. \\
\tt 0xB02 & MRW  &\tt minstret       & Machine instructions-retired counter. \\
\tt 0xB03 & MRW  &\tt mhpmcounter3   & Machine performance-monitoring counter. \\
\tt 0xB04 & MRW  &\tt mhpmcounter4   & Machine performance-monitoring counter. \\
& & \multicolumn{1}{c|}{\vdots} & \ \\
\tt 0xB1F & MRW  &\tt mhpmcounter31  & Machine performance-monitoring counter. \\
\tt 0xB80 & MRW  &\tt mcycleh        & Upper 32 bits of {\tt mcycle}, RV32 only. \\
\tt 0xB82 & MRW  &\tt minstreth      & Upper 32 bits of {\tt minstret}, RV32 only. \\
\tt 0xB83 & MRW  &\tt mhpmcounter3h  & Upper 32 bits of {\tt mhpmcounter3}, RV32 only. \\
\tt 0xB84 & MRW  &\tt mhpmcounter4h  & Upper 32 bits of {\tt mhpmcounter4}, RV32 only. \\
& & \multicolumn{1}{c|}{\vdots} & \ \\
\tt 0xB9F & MRW  &\tt mhpmcounter31h & Upper 32 bits of {\tt mhpmcounter31}, RV32 only. \\
\hline
\multicolumn{4}{|c|}{Machine Counter Setup \hl{机器计数器设置}} \\
\hline
\tt 0x320 & MRW  &\tt mcountinhibit  & Machine counter-inhibit register. \hl{机器计数器禁用寄存器}\\
\tt 0x323 & MRW  &\tt mhpmevent3     & Machine performance-monitoring event selector. \\
\tt 0x324 & MRW  &\tt mhpmevent4     & Machine performance-monitoring event selector. \\
& & \multicolumn{1}{c|}{\vdots} & \hl{机器性能监控事件选择器} \\
\tt 0x33F & MRW  &\tt mhpmevent31    & Machine performance-monitoring event selector. \\
\hline
\multicolumn{4}{|c|}{Debug/Trace Registers (shared with Debug Mode)} \\
\hline
\tt 0x7A0 & MRW &\tt tselect & Debug/Trace trigger register select. \\
\tt 0x7A1 & MRW &\tt tdata1 & First Debug/Trace trigger data register. \\
\tt 0x7A2 & MRW &\tt tdata2 & Second Debug/Trace trigger data register. \\
\tt 0x7A3 & MRW &\tt tdata3 & Third Debug/Trace trigger data register. \\
\tt 0x7A8 & MRW &\tt mcontext & Machine-mode context register. \\
\hline
\multicolumn{4}{|c|}{Debug Mode Registers } \\
\hline
\tt 0x7B0 & DRW &\tt dcsr & Debug control and status register. \\
\tt 0x7B1 & DRW &\tt dpc & Debug PC. \\
\tt 0x7B2 & DRW &\tt dscratch0 & Debug scratch register 0. \\
\tt 0x7B3 & DRW &\tt dscratch1 & Debug scratch register 1. \\
\hline
\end{tabular}
\end{center}
\caption{Currently allocated RISC-V machine-level CSR addresses. \\
\hl{现已分配空间的RISC-V机器级CSR地址
}}
\label{mcsrnames1}
\end{table}

\clearpage

\section{CSR Field Specifications \\
\hl{寄存器字段说明}}


The following definitions and abbreviations are used in specifying the
behavior of fields within the CSRs.

\hl{下面的定义和缩写用于规定控制和状态寄存器中各个字段的行为。}

\subsection*{Reserved Writes Preserve Values, Reads Ignore Values (WPRI)\\
\hl{写时保留不改变,读时忽略(WPRI)}}

Some whole read/write fields are reserved for future use.  Software
should ignore the values read from these fields, and should preserve
the values held in these fields when writing values to other fields of
the same register.
For forward compatibility, implementations that do not furnish these fields
must hardwire them to zero.
These fields are labeled \wpri\ in the register descriptions.

\begin{commentary}
To simplify the software model, any backward-compatible future
definition of previously reserved fields within a CSR must cope with
the possibility that a non-atomic read/modify/write sequence is used
to update other fields in the CSR.  Alternatively, the original CSR
definition must specify that subfields can only be updated atomically,
which may require a two-instruction clear bit/set bit sequence in
general that can be problematic if intermediate values are not legal.
\end{commentary}

\begin{commentary}
  \hl{
    为了简化软件模型,CSR中先前保留字段的任何向后兼容的未来的定义都必须应对
    使用非原子性的读/修改/写来更新CSR中其他字段的可能性。或者,原始的CSR定义
    必须指定子字段只能被原子性地更新,这通常可能需要一个双指令的清除位/设置位的
    序列,如果双指令的中间值不合法,则会产生问题。
  }
\end{commentary}

\subsection*{Write/Read Only Legal Values (WLRL) \\
\hl{只能读写合法值(WLRL}}

Some read/write CSR fields specify behavior for only a subset of
possible bit encodings, with other bit encodings reserved.  Software
should not write anything other than legal values to such a field, and
should not assume a read will return a legal value unless the last
write was of a legal value, or the register has not been written since
another operation (e.g., reset) set the register to a legal value.
These fields are labeled \wlrl\ in the register descriptions.

\hl{
  一些读/写CSR字段仅为可能的位编码子集指定行为,其他位编码保留。软件不应该
  向这样一个字段写入任何合法值以外的内容,并且不应该假设读取将返回一个合法值,
  除非最后一次写入是合法值,或者寄存器自从另一个操作(例如,重置)将寄存器
  设置为合法值后没有被写入。这些字段在寄存器描述中被标记为WLRL。
}

\begin{commentary}
Hardware implementations need only implement enough state bits to
differentiate between the supported values, but must always return the
complete specified bit-encoding of any supported value when read.
\end{commentary}

Implementations are permitted but not required to raise an illegal
instruction exception if an instruction attempts to write a
non-supported value to a \wlrl\ field.  Implementations can
return arbitrary bit patterns on the read of a \wlrl\ field when the last
write was of an illegal value, but the value returned should
deterministically depend on the illegal written value and
the value of the field prior to the write.

\subsection*{Write Any Values, Reads Legal Values (WARL) \\
\hl{
  可写任意值,只能读出合法值(WARL)
}}

Some read/write CSR fields are only defined for a subset of bit
encodings, but allow any value to be written while guaranteeing to
return a legal value whenever read.  Assuming that writing the CSR has
no other side effects, the range of supported values can be determined
by attempting to write a desired setting then reading to see if the
value was retained.  These fields are labeled \warl\ in the register
descriptions.

Implementations will not raise an exception on writes of unsupported
values to a \warl\ field.  Implementations can
return any legal value on the read of a \warl\ field when the last
write was of an illegal value, but the legal value returned should
deterministically depend on the illegal written value and
the architectural state of the hart.

\section{CSR Width Modulation \\
\hl{CSR宽度调整策略}}
\label{sec:csrwidthmodulation}

If the width of a CSR is changed (for example, by changing MXLEN or UXLEN, as
described in Section~\ref{xlen-control}), the values of the {\em writable}
fields and bits of the new-width CSR are, unless specified otherwise,
determined from the previous-width CSR as though by this algorithm:

\begin{enumerate}

\item The value of the previous-width CSR is copied to a temporary register of
the same width.

\item For the read-only bits of the previous-width CSR, the bits at the same
positions in the temporary register are set to zeros.

\item The width of the temporary register is changed to the new width. If the
new width $W$ is narrower than the previous width, the least-significant $W$
bits of the temporary register are retained and the more-significant bits are
discarded. If the new width is wider than the previous width, the temporary
register is zero-extended to the wider width.

\item Each writable field of the new-width CSR takes the value of the bits at
the same positions in the temporary register.

\end{enumerate}

Changing the width of a CSR is not a read or write of the CSR and thus
does not trigger any side effects.



\input{machine}
%\input{supervisor}
%\input{hypervisor}
%\input{n}
%\input{priv-insns}

%\input{priv-history}

\bibliographystyle{plain}
\bibliography{riscv-spec}

\end{document}
